% !TEX TS-program = xelatex
% !TEX encoding = UTF-8 Unicode
% -*- coding: UTF-8; -*-
% vim: set fenc=utf-8

%%%%%%%%%%%%%%%%%%%%%%%%%%%%%%%%%%%%%%%%%%%%%%%%%%%%%%%%%%%%%%%%%
%% SIMPLE-RESUME-CV
%% <https://github.com/zachscrivena/simple-resume-cv>
%% This is free and unencumbered software released into the
%% public domain; see <http://unlicense.org> for details.
%%%%%%%%%%%%%%%%%%%%%%%%%%%%%%%%%%%%%%%%%%%%%%%%%%%%%%%%%%%%%%%%%

\documentclass[letterpaper,MMMyyyy,nonstopmode]{simpleresumecv}
% Class options:
% a4paper, letterpaper, nonstopmode, draftmode
% MMMyyyy, ddMMMyyyy, MMMMyyyy, ddMMMMyyyy, yyyyMMdd, yyyyMM, yyyy

%%%%%%%%%%%%%%%%%%%%%%%%%%%%%%%%%%%%%%%%%%%%%%%%%%%%%%%%%%%%%%%%%
%% PREAMBLE.
%%%%%%%%%%%%%%%%%%%%%%%%%%%%%%%%%%%%%%%%%%%%%%%%%%%%%%%%%%%%%%%%%

% CV Info (to be customized).
\newcommand{\ResumeAuthor}{James Hardjadinata}
\newcommand{\ResumeTitle}{Resume}
\newcommand{\ResumeNote}{Resume compiled on {\today}}
\newcommand{\ResumeWebsite}{github.com/SVT125}
\newcommand{\ResumeLinkedin}{linkedin.com/in/jameshardjadinata}
\newcommand{\ResumeEmail}{jhardjadinata14@gmail.com}

% PDF settings and properties.
\hypersetup{
pdftitle={\ResumeTitle},
pdfauthor={\ResumeAuthor},
pdfsubject={\ResumeWebsite},
pdfcreator={XeLaTeX},
pdfproducer={},
pdfkeywords={},
unicode=true,
bookmarks=true,
bookmarksopen=true,
pdfstartview=FitH,
pdfpagelayout=OneColumn,
pdfpagemode=UseOutlines,
hidelinks,
breaklinks}

% Shorthand.
\newcommand{\Code}[1]{\mbox{\textbf{#1}}}
\newcommand{\CodeCommand}[1]{\mbox{\textbf{\textbackslash{#1}}}}

%%%%%%%%%%%%%%%%%%%%%%%%%%%%%%%%%%%%%%%%%%%%%%%%%%%%%%%%%%%%%%%%%
%% ACTUAL DOCUMENT.
%%%%%%%%%%%%%%%%%%%%%%%%%%%%%%%%%%%%%%%%%%%%%%%%%%%%%%%%%%%%%%%%%

\begin{document}

%%%%%%%%%%%%%%%
% TITLE BLOCK %
%%%%%%%%%%%%%%%

\Title{\ResumeAuthor}
\begin{SubTitle}
(949)\,232-3835
\par
\href{mailto:\ResumeEmail}
{\ResumeEmail}
\,\textbullet\,
\href{\ResumeLinkedin}
{\url{\ResumeLinkedin}}\,
\,\textbullet\,
\href{\ResumeWebsite}
{\url{\ResumeWebsite}}
\end{SubTitle}

\begin{Body}

%%%%%%%%%%%%%%%
%% EDUCATION %%
%%%%%%%%%%%%%%%

\Section
{Education}
{Education}
{PDF:Education}

\Entry
\href{https://uci.edu/}
{\textbf{University of California, Irvine}},
Irvine, CA
\hfill
\DatestampYMD{2014}{10}{23} --
\DatestampYMD{2017}{03}{24}
\Gap
B.S., Computer Science, GPA 3.9\newline

\Entry
\href{http://education.oracle.com/pls/web_prod-plq-dad/db_pages.getpage?page_id=5001&get_params=p_exam_id:1Z0-803&p_org_id=&lang=}
{\textbf{Oracle Certified Associate, Java SE 7 Programmer}},
Oracle Corporation
\hfill
\DatestampYMD{2013}{04}{30}
\newline
\Gap

\hrule

%%%%%%%%%%%%%%%%
%% EXPERIENCE %%
%%%%%%%%%%%%%%%%

\Section
{Experience}
{Experience}
{PDF:Experience}

\Entry
\href{http://www.clustrix.com/}
{\textbf{Software Test Engineer Intern at Clustrix}},
San Francisco, CA
\hfill
\DatestampYMD{2016}{06}{27} --
\DatestampYMD{2016}{08}{27}
\Gap
Wrote automated test cases (regression, stress, acceptance, etc.) in Python and bash involving extensive SQL/Clustrix-exclusive
SQL; reported bugs and reviewed code for the next version release, especially relating to replication, rebalancing, user-level locks, and cluster reporting. Added features to internal code for both performance and QA
testing. Responsible for knowledge of the product in assisting clients.\newline

\Entry
\href{http://www.theportal.io/}
{\textbf{Software Development Intern at The Portal by K5 Ventures}},
Irvine, CA
\hfill
\DatestampYMD{2015}{10}{20} --
\DatestampYMD{2016}{06}{21}
\Gap
Developed native Android apps, MEAN stack RESTful web apps, and project consulting as a team lead/member, taking full
responsibility for clients and their projects. Accounted for daily decision making and timely project delivery with Agile processes.\newline

\hrule

%%%%%%%%%%%%
%% SKILLS %%
%%%%%%%%%%%%

\Section
{Skills}
{Skills}
{PDF:Skills}

\begin{flushleft}
\textbf{Programming Languages:} Java, C, C++, C\#, Python, HTML/CSS, Javascript
\textbf{Technologies/Applications:} Linux, SQL, MongoDB, Node.js, Unity, Git, Mercurial\newline
\end{flushleft}

\hrule

%%%%%%%%%%%%%%%%%%%%%%%
%% PROJECTS & AWARDS %%
%%%%%%%%%%%%%%%%%%%%%%%

\Section
{Projects \& Awards}
{Projects \& Awards}
{PDF:ProjectsAwards}

\Entry
\BulletItem
\href{https://github.com/SVT125/dingus-bot}
{\textbf{Discord Bot (2017)}} -- A Discord bot  written in Python using Django as a backend and incorporates many APIs for its 20+ commands e.g. Discord.py, Google, Reddit, etc. Commands include  music playback w/ search querying, info retrieval from popular sites like Imgur, and nifty features like Markdown message formatting.\newline

\Entry
\BulletItem
\href{https://github.com/trizzle21/Datathon2016}
{\textbf{CellTowers, Machine Learning Award @ UCI Mobile Data Science Hackathon (2016)}} -- Developed a Python application for visualizing cell tower data, predicting data usage and other variables using PCA and Kalman filtering, and provides heat maps for optimization of tower signals. Worked on parsing and projecting the data, PCA, and graphing using pandas and matplotlib.\newline

\Entry
\BulletItem
\href{https://github.com/SVT125/HackUCI2015}
{\textbf{Friendboard, HackUCI (2015)}} -- Created an Android app with teammates that transcribes speech phrases to play back after phone calls. Uses the Microsoft Project Oxford speech recognition API and other APIs for audio normalization. Handled the Microsoft API, wrote all Android code, and integrated components of the project together.\newline

\Entry
\BulletItem
\href{https://github.com/SVT125/GermanBridge}
{\textbf{Card Suite (2015)}} -- Native Android app written with friends which houses 3 card games  under one app with a simple, clean design; designed the game state structure, programmed the XML layout and minimax socially-driven AI sourced from research papers; wrote most of the backing code and handled publishing.\newline

\Entry
\BulletItem
\href{https://github.com/brendonwai/PAS}
{\textbf{Guesstimate, 1st @ IEEE GameSig/Semifinalist @ Microsoft Imagine Cup (2014-2015)}} -- Created a Unity/C\# game in a group game jam that won 1st place against 30+ games from several universities for simplicity, marketability, and cost. Wrote the backing difficulty algorithm, class/prefab implementations.\newline

\Entry
\BulletItem
{\textbf{Flashcards Lab (2014)}} -- Developed an Android lab for students to implement for a Java class at UCI, structured to cover basic Java syntax and OOP fundamentals; utilized various Android APIs and SQL deployed on an Amazon AWS server to store player statistics.\newline

\Entry
\BulletItem
\href{https://github.com/SVT125/Research-CS199}
{\textbf{Spiral Galaxy Research (2014)}} -- Group research for finding outliers in spiral galaxies, used Python + Java to retrieve the data values of thousands of galaxies and apply various mathematical/machine learning algorithms to determine the sets of variables that identified outlier galaxies; implemented the algorithms involved and some data parsing.\newline

\end{Body}
\end{document}